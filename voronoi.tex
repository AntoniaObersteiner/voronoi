\documentclass{article}
\usepackage{amsmath}
\usepackage{amssymb}


\newcommand\point[1][\pointindex]{{{P}_{#1}}}
\newcommand\points{\point[\pointindices]}
\newcommand\pointindex[1][]{{{p}_{#1}}}
\newcommand\pointindices{\Pi}
\newcommand\ooint[1][\oointindex]{\point[#1]}
\newcommand\oointindex[1][]{{{o}_{#1}}}

\newcommand\cnter[1][\cnterindex]{Q_{#1}}
\newcommand\cnters{\cnter[\cnterindices]}
\newcommand\cnterindex[1][]{{q_{#1}}}
\newcommand\cnterindices{\Omega}

\newcommand\relevant[1][\relevantindex]{Q_{#1}}
\newcommand\relevants{\relevant[\relevantindices]}
\newcommand\relevantindex[1][]{{r_{#1}}}
\newcommand\relevantindices{R}
\newcommand\selevant[1][\selevantindex]{\relevant[#1]}
\newcommand\selevantindex[1][]{{s_{#1}}}

\newcommand\edge[1][\edgeindex]{E_{#1}}
\newcommand\edges{\edge[\edgeindices]}
\newcommand\edgeindex[1][]{{r_{#1}}}
\newcommand\edgeindices{}
\newcommand\edgepoint[1][\of{\relevantindex, \selevantindex}]{V_{#1}}

\DeclareMathOperator{\Encode}{\omega}
\DeclareMathOperator{\Decode}{\pi}
\newcommand\encode[1]{\Encode\of{#1}}
\newcommand\decode[1]{\Decode\of{#1}}
\newcommand\triple[1]{#1[0], #1[1], #1[2]}
\newcommand\texttriple[1]{$#1[0], #1[1]$ and $#1[2]$}

\newcommand\nats{\mathbb N}
\newcommand\reals{\mathbb R}

\newcommand\of   [1]{\left(  {#1}\right)}
\newcommand\set  [1]{\left\{ {#1}\right\}}
\newcommand\abs  [1]{\left|  {#1}\right|}
\newcommand\at   [1]{\left[  {#1}\right]}
\newcommand\oival[1]{\left[\,{#1}\right)}


\author{Antonia Obersteiner}
\title{The Voronoi Graph}

\begin{document}

\section{Exposition}
\paragraph{The Points} $\points := \set{\point[\pointindex] \in\reals^2}$ have
indices $\pointindex \in \pointindices := \oival{0, \abs{\pointindices}}\cap\nats$.
The Points are pairwise distinct.
\paragraph{The Centers} $\cnters := \set{\cnter[\cnterindex] \in\reals^2}$ have
indices $\cnterindex \in \cnterindices := \oival{0, \abs{\cnterindices}}\cap\nats$.
Each Center $\cnter[\cnterindex]$ is equally distant from three Points
	$\point[{\pointindex[0]}]$,
	$\point[{\pointindex[1]}]$ and
	$\point[{\pointindex[2]}]$,
with Encoding $\encode{\cdot}$ and Decoding $\decode{\cdot}$ where
$\cnterindex = \encode{\triple{\pointindex}}$ such that
$\decode{\encode{\triple{\pointindex}}} := \of{\triple{\pointindex}}$.
Thus, $\abs{\cnterindices} = \abs{\pointindices}^3$.
Note that the order of \texttriple{\pointindex} does not matter,
so a lot of the $\cnter$ will be identical.
Non-distinct \texttriple{\pointindex} are not relevant to the problem and their
$\cnter[\encode{\triple{\pointindex}}]$ are therefore undefined.
Therefore, $\pointindex[0] > \pointindex[1] > \pointindex[2]$ will usually hold.
\end{document}
